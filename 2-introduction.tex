\section{Introduction}
A key source of frustration with electronic medical record (EMR) systems stems from the inability to retrieve relevant patient information efficiently \cite{Nolan2017,Roman2017,Yang2011,Natarajan2010}. Current EMR systems do not possess sophisticated search capability nor do they prioritize patient information relative to the clinical task at hand \cite{Ruppel2020,Mazur2019}. The inability to identify relevant patient information can lead to poor care and medical errors \cite{Hall2004,Ahmed2011,Pollack2020}. Further, in complex clinical environments, such as the intensive care unit (ICU), large quantities of data per patient accumulate rapidly \cite{Manor-Shulman2008}, which can exacerbate information retrieval challenges. EMR systems that prioritize the display of relevant patient information are therefore needed to minimize the time and effort that physicians spend in identifying relevant information.

Various solutions have been proposed for effective prioritization and display of patient information in EMR systems \cite{Law2005, Koch2013,Wright2019,Pickering2015}, most of which are based on rules that have been developed to customize and organize the display of patient information. In contrast to rule-based approaches, we developed and evaluated a data-driven approach called the learning EMR (LEMR) system in a prior study \cite{King2018, King2019}. The LEMR system tracks physician information-seeking behavior and uses it to learn machine learning models that predict which information is relevant in a given clinical context. Those predictions are used to highlight the relevant data in the EMR system to draw a physician’s attention.

However, information-seeking behavior has been shown to vary across individual physicians as well as across EMR system user types such as physicians, nurses, and pharmacists \cite{Nolan2017,Ruppel2020}. In this study, we use hierarchical models to explicitly model this variability because such models have been shown to be useful when the data are collected from subjects with different behaviors \cite{gelman2006data}. In particular, we compare the performance of hierarchical logistic regression models and standard logistic regression models in predicting relevant patient information in a LEMR system.

The remainder of this paper is organized as follows. In the Background section, we review the LEMR system, briefly describe hierarchical models, and describe prior work on physician-related variability. In the Methods section, we describe the data collection and preparation, the experimental details, and the evaluation measures. We present the results of the experiments in the Results section, and close with Discussion and Conclusion sections.
