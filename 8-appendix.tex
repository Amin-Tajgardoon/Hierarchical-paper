\appendix
\section*{Appendix}
\section{Hierarchical logistic regression (HLR) models}\label{sec:app}
Hierarchical logistic regression (HLR) models are a generalization of standard logistic regression (LR) models in which distinct logistic regression models are fit at different levels of hierarchically structured data \cite{gelman2006data}. In the context of the LEMR system, we define an HLR model as follows (a boldface character represents a matrix or vector of parameters or values). Let $D=\{\boldsymbol{X}, \boldsymbol{y}\}$ be a LEMR data set where $\boldsymbol{X}$ denotes a set of $N$ patient cases in the EMR, each with $K$ predictor variables including demographics, medication administrations, laboratory test results, and vital signs. Let $\boldsymbol{y}$ represent an EMR data item for $N$ patient cases with binary values of \textit{relevant} and \textit{not relevant}. The data is reviewed by $J$ physicians, each reviewing $n_j$ cases such that $\sum_{j=1}^{J}{n_j}=N$. We formulate an HLR model as follows: 
\begin{equation}\label{eq:2}
\begin{split}
    y & \sim Binomial(\boldsymbol{p}, N), \\
    \boldsymbol{p} = logit^{-1} & (\beta_0 + \boldsymbol{\beta}\boldsymbol{X} + \boldsymbol{\phi_0} + \boldsymbol{\phi} \boldsymbol{Z})
\end{split}
\end{equation}

\begin{equation}\label{eq:3}
\begin{split}
    \beta_k & \sim Normal(0, \sigma_{\beta}^2), \qquad j= 1,..., J\\
    \phi_{k_j} & \sim Normal(0, \sigma_{\phi}^2),  \qquad k= 0,..., K
\end{split}
\end{equation}

\begin{equation}\label{eq:4}
    \sigma_{\phi} \sim Half Cauchy(0, \tau_{\phi})
\end{equation}
where $\beta_0$ and $\boldsymbol{\beta}$ are population-level intercept and coefficients, and $\boldsymbol{\phi_0}$ and $\boldsymbol{\phi}$ denote physician-level intercept and coefficients. $\boldsymbol{Z}$ corresponds to the physician-level design matrix, which is a sparse expansion of $\boldsymbol{X}$ with $N$ rows and $J\times K$ columns, splitting $\boldsymbol{X}$ into $J$ segments. Below is a simple example of $\boldsymbol{X}$ and $\boldsymbol{Z}$ matrices with  $N=5$ patients reviewed by $J=2$ physicians (colored in blue and red) and $K=3$ predictor variables:

\begin{equation*}
   \boldsymbol{X} =	\begin{bmatrix}
\color{blue} 0 & \color{blue} 0 & \color{blue} 1\\
\color{blue} 0 & \color{blue} 1 & \color{blue} 0\\
\color{red} 1 & \color{red} 1 & \color{red} 0\\
\color{red} 0 & \color{red} 1 & \color{red} 1\\

\color{red} 1 & \color{red} 1 & \color{red} 0 \end{bmatrix} , \qquad \boldsymbol{Z} =	\begin{bmatrix}
\color{blue} 0 & \color{blue} 0 & \color{blue} 1 & 0 & 0 & 0\\
\color{blue} 0 & \color{blue} 1 & \color{blue} 0 & 0 & 0 & 0\\
0 & 0 & 0 & \color{red} 1 & \color{red} 1 & \color{red} 0\\
0 & 0 & 0 & \color{red} 0 & \color{red} 1 & \color{red} 1\\
0 & 0 & 0 & \color{red} 1 & \color{red} 1 & \color{red} 0 
\end{bmatrix}
\end{equation*}

In Equation~\ref{eq:3}, $\beta_k$ and $\phi_k$ denote the coefficients of the $k^{\text{th}}$ predictor variable ($k=0$ denotes the intercept) for the population and physician $j$’s models, and are assumed a priori to be centered at 0 with standard deviations $\sigma_{\beta}$ and $\sigma_{\phi}$, respectively. We define the prior for $\sigma_{\phi}$ as a Cauchy distribution with scale parameter $\tau_{\phi}$, and restrict it to positive values (hence the \textit{half} Cauchy). $\sigma_{\beta}$ and $\tau_{\phi}$ are the model hyperparameters and are tuned in the model training phase. Markov chain Monte Carlo (MCMC) methods are used to estimate the parameters’ posterior distributions. MCMC methods draw samples sequentially from the posterior distribution and improve the draws at each step to better approximate the distribution.

\section{Cross validation details}\label{sec:app_B}
Each model was trained and evaluated independently in a stratified 10-fold cross validation setting. The following preprocessing steps were applied to the training and test sets at each iteration:

\begin{enumerate}
    \item \textit{Imputation}. Missing values in predictor variables were imputed by the median or mode for continuous and discrete variables, respectively. The imputed value was derived from the training set and applied to both the training and test sets.
    
    \item \textit{Feature selection}. We used supervised univariate feature selection to reduce the number of features before deriving models. In particular, we used analysis of variance (ANOVA) and Fisher’s exact tests (significance level = 0.01) for continuous and binary predictor variables, respectively. We limited the predictors to a maximum of 100 variables. The feature selection was performed on the training set and then applied to both the training and test sets.
    
    \item \textit{Feature standardization}. Continuous predictor variables were rescaled to be centered at zero and have a unit standard deviation. We calculated the mean and standard deviation statistics from the training set and used them to standardize both training and test sets.
\end{enumerate}

Each model was trained on the training set and evaluated on the test set. Model hyperparameters were tuned in an inner stratified 3-fold cross validation of the training set. In particular, for HLR models, we selected the best values of the hyperparameters (i.e., $\sigma_{\beta}$ and $\tau_{\phi}$ in Equations~\ref{eq:3} and \ref{eq:4} above) from the set of values $\{0.01, 0.1, 1, 5, 10\}$. For LR models, LASSO regularization was used and the optimal regularization parameter ($\lambda$) was chosen from an automatically generated sequence of 100 values as described in \cite{Friedman2010}.

\section{Tables and figures}\label{sec:app_C}


% \vspace{5mm}

\begin{scriptsize}
\begin{center}
\setlength{\aboverulesep}{0.1pt}
    \setlength{\belowrulesep}{0.1pt}
    \renewcommand{\arraystretch}{1.2}
    \setcounter{table}{0} \renewcommand{\thetable}{A\arabic{table}}
\begin{longtable}{|ccc|c|c|c|c|c|c|}
    \caption{Performance of predictive models for 73 target variables. Higher AUROC and AUPRC denotes better discrimination and lower ECE denotes better calibration. Relevant/Available: number of cases in which the target variable was annotated as relevant over number of cases for which the target variable was measured in the EMR and was available for annotation. Domain: IO=input/output, L=laboratory test result, M=medication, V=vital sign, VS=ventilator setting.} \label{tab:s1}\\
\toprule
\multicolumn{1}{|c|}{\multirow{2}[4]{*}{\textbf{Target variable}}} & \multicolumn{1}{c|}{\multirow{2}[4]{*}{\textbf{Domain}}} & \textbf{Relevant/} & \multicolumn{2}{c|}{\textbf{AUROC}} & \multicolumn{2}{c|}{\textbf{AUPRC}} & \multicolumn{2}{c|}{\textbf{ECE}} \\
\cmidrule{4-9}\multicolumn{1}{|c|}{} & \multicolumn{1}{c|}{} & \textbf{Available} & \textbf{HLR} & \textbf{LR} & \textbf{HLR} & \textbf{LR} & \textbf{HLR} & \textbf{LR} \\
\midrule
\multicolumn{1}{|c|}{acetaminophen} & \multicolumn{1}{c|}{M} & 12/68 & 0.67  & 0.51  & 0.31  & 0.22  & 0.14  & 0.2 \\
\midrule
\multicolumn{1}{|c|}{albumin} & \multicolumn{1}{c|}{L} & 19/107 & 0.86  & 0.84  & 0.49  & 0.54  & 0.07  & 0.08 \\
\midrule
\multicolumn{1}{|c|}{albuterol ipratropium} & \multicolumn{1}{c|}{M} & 15/59 & 0.71  & 0.87  & 0.49  & 0.77  & 0.15  & 0.08 \\
\midrule
\multicolumn{1}{|c|}{ALT (SGPT)} & \multicolumn{1}{c|}{L} & 23/104 & 0.78  & 0.71  & 0.57  & 0.45  & 0.11  & 0.17 \\
\midrule
\multicolumn{1}{|c|}{ammonia} & \multicolumn{1}{c|}{L} & 12/39 & 0.65  & 0.51  & 0.47  & 0.35  & 0.14  & 0.26 \\
\midrule
\multicolumn{1}{|c|}{ampicillin sulbactam} & \multicolumn{1}{c|}{M} & 9/20 & 0.57  & 0.48  & 0.55  & 0.6   & 0.39  & 0.47 \\
\midrule
\multicolumn{1}{|c|}{anion gap} & \multicolumn{1}{c|}{L} & 19/111 & 0.87  & 0.75  & 0.69  & 0.54  & 0.08  & 0.08 \\
\midrule
\multicolumn{1}{|c|}{aspirin} & \multicolumn{1}{c|}{M} & 15/45 & 0.63  & 0.7   & 0.54  & 0.71  & 0.13  & 0.17 \\
\midrule
\multicolumn{1}{|c|}{AST (SGOT)} & \multicolumn{1}{c|}{L} & 25/106 & 0.71  & 0.6   & 0.47  & 0.36  & 0.13  & 0.19 \\
\midrule
\multicolumn{1}{|c|}{band cell count} & \multicolumn{1}{c|}{L} & 13/81 & 0.69  & 0.72  & 0.38  & 0.39  & 0.13  & 0.09 \\
\midrule
\multicolumn{1}{|c|}{base solution} & \multicolumn{1}{c|}{M} & 50/81 & 0.63  & 0.62  & 0.73  & 0.65  & 0.2   & 0.22 \\
\midrule
\multicolumn{1}{|c|}{bicarbonate} & \multicolumn{1}{c|}{L} & 103/167 & 0.79  & 0.7   & 0.83  & 0.8   & 0.11  & 0.14 \\
\midrule
\multicolumn{1}{|c|}{bicarbonate (HCO3), arterial} & \multicolumn{1}{c|}{L} & 11/102 & 0.63  & 0.74  & 0.19  & 0.35  & 0.09  & 0.06 \\
\midrule
\multicolumn{1}{|c|}{bilirubin, direct} & \multicolumn{1}{c|}{L} & 16/82 & 0.75  & 0.6   & 0.64  & 0.27  & 0.1   & 0.16 \\
\midrule
\multicolumn{1}{|c|}{bilirubin, total} & \multicolumn{1}{c|}{L} & 36/103 & 0.82  & 0.78  & 0.65  & 0.66  & 0.13  & 0.17 \\
\midrule
\multicolumn{1}{|c|}{blood urea nitrogen (BUN)} & \multicolumn{1}{c|}{L} & 114/166 & 0.76  & 0.66  & 0.83  & 0.8   & 0.14  & 0.16 \\
\midrule
\multicolumn{1}{|c|}{calcium} & \multicolumn{1}{c|}{L} & 41/155 & 0.76  & 0.79  & 0.56  & 0.54  & 0.11  & 0.1 \\
\midrule
\multicolumn{1}{|c|}{central venous pressure (CVP)} & \multicolumn{1}{c|}{V} & 31/103 & 0.76  & 0.59  & 0.63  & 0.38  & 0.11  & 0.24 \\
\midrule
\multicolumn{1}{|c|}{chlorhexidine topical} & \multicolumn{1}{c|}{M} & 20/86 & 0.88  & 0.81  & 0.71  & 0.63  & 0.13  & 0.11 \\
\midrule
\multicolumn{1}{|c|}{chloride} & \multicolumn{1}{c|}{L} & 106/167 & 0.85  & 0.77  & 0.88  & 0.81  & 0.09  & 0.11 \\
\midrule
\multicolumn{1}{|c|}{dextrose 5\% in water} & \multicolumn{1}{c|}{M} & 17/46 & 0.63  & 0.55  & 0.6   & 0.46  & 0.29  & 0.33 \\
\midrule
\multicolumn{1}{|c|}{docusate} & \multicolumn{1}{c|}{M} & 9/47 & 0.85  & 0.82  & 0.66  & 0.59  & 0.15  & 0.08 \\
\midrule
\multicolumn{1}{|c|}{famotidine} & \multicolumn{1}{c|}{M} & 26/77 & 0.75  & 0.7   & 0.57  & 0.55  & 0.12  & 0.19 \\
\midrule
\multicolumn{1}{|c|}{fentanyl} & \multicolumn{1}{c|}{M} & 18/83 & 0.66  & 0.59  & 0.41  & 0.39  & 0.14  & 0.22 \\
\midrule
\multicolumn{1}{|c|}{fraction of inspired oxygen (FiO2)} & \multicolumn{1}{c|}{VS} & 95/142 & 0.72  & 0.69  & 0.83  & 0.81  & 0.13  & 0.18 \\
\midrule
\multicolumn{1}{|c|}{furosemide} & \multicolumn{1}{c|}{M} & 28/66 & 0.65  & 0.66  & 0.57  & 0.56  & 0.15  & 0.25 \\
\midrule
\multicolumn{1}{|c|}{glucose} & \multicolumn{1}{c|}{L} & 114/164 & 0.66  & 0.64  & 0.8   & 0.81  & 0.17  & 0.2 \\
\midrule
\multicolumn{1}{|c|}{hemoglobin} & \multicolumn{1}{c|}{L} & 123/156 & 0.76  & 0.57  & 0.9   & 0.82  & 0.08  & 0.16 \\
\midrule
\multicolumn{1}{|c|}{heparin} & \multicolumn{1}{c|}{M} & 38/97 & 0.68  & 0.71  & 0.52  & 0.6   & 0.14  & 0.24 \\
\midrule
\multicolumn{1}{|c|}{hydrocortisone} & \multicolumn{1}{c|}{M} & 10/20 & 0.35  & 0.28  & 0.44  & 0.4   & 0.3   & 0.54 \\
\midrule
\multicolumn{1}{|c|}{input/output (I/O)} & \multicolumn{1}{c|}{IO} & 81/167 & 0.8   & 0.74  & 0.83  & 0.72  & 0.13  & 0.14 \\
\midrule
\multicolumn{1}{|c|}{INR} & \multicolumn{1}{c|}{L} & 62/118 & 0.7   & 0.65  & 0.7   & 0.63  & 0.1   & 0.22 \\
\midrule
\multicolumn{1}{|c|}{insulin aspartate (Novolog)} & \multicolumn{1}{c|}{M} & 11/27 & 0.36  & 0.36  & 0.34  & 0.36  & 0.45  & 0.45 \\
\midrule
\multicolumn{1}{|c|}{insulin glargine (Lantus)} & \multicolumn{1}{c|}{M} & 13/21 & 0.35  & 0.21  & 0.6   & 0.48  & 0.37  & 0.6 \\
\midrule
\multicolumn{1}{|c|}{insulin regular (Humulin R, Novolin R)} & \multicolumn{1}{c|}{M} & 36/77 & 0.61  & 0.55  & 0.65  & 0.56  & 0.21  & 0.34 \\
\midrule
\multicolumn{1}{|c|}{ionized Ca} & \multicolumn{1}{c|}{L} & 30/122 & 0.72  & 0.62  & 0.49  & 0.35  & 0.11  & 0.23 \\
\midrule
\multicolumn{1}{|c|}{lactate} & \multicolumn{1}{c|}{L} & 50/109 & 0.73  & 0.68  & 0.72  & 0.59  & 0.1   & 0.21 \\
\midrule
\multicolumn{1}{|c|}{lactulose} & \multicolumn{1}{c|}{M} & 9/16 & 0.92  & 0.67  & 0.95  & 0.74  & 0.19  & 0.33 \\
\midrule
\multicolumn{1}{|c|}{lansoprazole} & \multicolumn{1}{c|}{M} & 9/34 & 0.73  & 0.65  & 0.55  & 0.47  & 0.2   & 0.24 \\
\midrule
\multicolumn{1}{|c|}{levetiracetam} & \multicolumn{1}{c|}{M} & 9/16 & 0.93  & 0.79  & 0.94  & 0.88  & 0.24  & 0.22 \\
\midrule
\multicolumn{1}{|c|}{lorazepam} & \multicolumn{1}{c|}{M} & 9/37 & 0.35  & 0.43  & 0.34  & 0.26  & 0.34  & 0.25 \\
\midrule
\multicolumn{1}{|c|}{magnesium} & \multicolumn{1}{c|}{L} & 73/163 & 0.77  & 0.72  & 0.71  & 0.65  & 0.13  & 0.13 \\
\midrule
\multicolumn{1}{|c|}{metoprolol} & \multicolumn{1}{c|}{M} & 19/57 & 0.52  & 0.43  & 0.36  & 0.35  & 0.24  & 0.38 \\
\midrule
\multicolumn{1}{|c|}{metronidazole} & \multicolumn{1}{c|}{M} & 16/28 & 0.67  & 0.59  & 0.76  & 0.67  & 0.22  & 0.33 \\
\midrule
\multicolumn{1}{|c|}{midazolam} & \multicolumn{1}{c|}{M} & 9/51 & 0.62  & 0.59  & 0.3   & 0.3   & 0.14  & 0.19 \\
\midrule
\multicolumn{1}{|c|}{mixed venous oxygen saturation (SvO2)} & \multicolumn{1}{c|}{V} & 9/39 & 0.64  & 0.47  & 0.54  & 0.32  & 0.21  & 0.23 \\
\midrule
\multicolumn{1}{|c|}{mode} & \multicolumn{1}{c|}{VS} & 71/140 & 0.75  & 0.66  & 0.73  & 0.66  & 0.11  & 0.2 \\
\midrule
\multicolumn{1}{|c|}{neutrophils} & \multicolumn{1}{c|}{L} & 24/146 & 0.75  & 0.68  & 0.39  & 0.35  & 0.09  & 0.13 \\
\midrule
\multicolumn{1}{|c|}{norepinephrine} & \multicolumn{1}{c|}{M} & 17/35 & 0.51  & 0.47  & 0.48  & 0.56  & 0.25  & 0.35 \\
\midrule
\multicolumn{1}{|c|}{oxygen saturation (SaO2), arterial} & \multicolumn{1}{c|}{V} & 103/166 & 0.77  & 0.73  & 0.83  & 0.78  & 0.15  & 0.15 \\
\midrule
\multicolumn{1}{|c|}{pantoprazole} & \multicolumn{1}{c|}{M} & 16/42 & 0.79  & 0.81  & 0.73  & 0.69  & 0.15  & 0.15 \\
\midrule
\multicolumn{1}{|c|}{partial pressure of carbon dioxide (PaCO2), arterial} & \multicolumn{1}{c|}{L} & 31/130 & 0.69  & 0.64  & 0.41  & 0.32  & 0.09  & 0.21 \\
\midrule
\multicolumn{1}{|c|}{partial pressure of oxygen (PaO2), arterial} & \multicolumn{1}{c|}{L} & 30/129 & 0.85  & 0.7   & 0.68  & 0.41  & 0.08  & 0.16 \\
\midrule
\multicolumn{1}{|c|}{PEEP} & \multicolumn{1}{c|}{VS} & 9/22 & 0.84  & 0.87  & 0.8   & 0.87  & 0.26  & 0.17 \\
\midrule
\multicolumn{1}{|c|}{pH, arterial} & \multicolumn{1}{c|}{L} & 46/129 & 0.67  & 0.63  & 0.54  & 0.44  & 0.17  & 0.22 \\
\midrule
\multicolumn{1}{|c|}{phosphate} & \multicolumn{1}{c|}{L} & 69/160 & 0.77  & 0.74  & 0.75  & 0.62  & 0.13  & 0.19 \\
\midrule
\multicolumn{1}{|c|}{piperacillin tazobactam} & \multicolumn{1}{c|}{M} & 24/47 & 0.81  & 0.67  & 0.85  & 0.67  & 0.2   & 0.27 \\
\midrule
\multicolumn{1}{|c|}{platelets} & \multicolumn{1}{c|}{L} & 116/156 & 0.8   & 0.65  & 0.89  & 0.83  & 0.08  & 0.17 \\
\midrule
\multicolumn{1}{|c|}{potassium} & \multicolumn{1}{c|}{L} & 120/167 & 0.73  & 0.67  & 0.87  & 0.84  & 0.12  & 0.15 \\
\midrule
\multicolumn{1}{|c|}{potassium chloride} & \multicolumn{1}{c|}{M} & 28/125 & 0.84  & 0.78  & 0.66  & 0.58  & 0.07  & 0.09 \\
\midrule
\multicolumn{1}{|c|}{propofol} & \multicolumn{1}{c|}{M} & 17/42 & 0.45  & 0.5   & 0.49  & 0.45  & 0.34  & 0.35 \\
\midrule
\multicolumn{1}{|c|}{PTT} & \multicolumn{1}{c|}{L} & 15/101 & 0.57  & 0.63  & 0.24  & 0.25  & 0.13  & 0.16 \\
\midrule
\multicolumn{1}{|c|}{respiratory rate (RR)} & \multicolumn{1}{c|}{V} & 121/167 & 0.7   & 0.64  & 0.85  & 0.82  & 0.15  & 0.2 \\
\midrule
\multicolumn{1}{|c|}{senna} & \multicolumn{1}{c|}{M} & 10/43 & 0.8   & 0.85  & 0.55  & 0.55  & 0.17  & 0.12 \\
\midrule
\multicolumn{1}{|c|}{sodium (Na)} & \multicolumn{1}{c|}{L} & 128/167 & 0.72  & 0.57  & 0.89  & 0.81  & 0.11  & 0.24 \\
\midrule
\multicolumn{1}{|c|}{sodium chloride 0.9\%} & \multicolumn{1}{c|}{M} & 65/144 & 0.62  & 0.57  & 0.56  & 0.53  & 0.19  & 0.28 \\
\midrule
\multicolumn{1}{|c|}{temperature} & \multicolumn{1}{c|}{V} & 144/167 & 0.72  & 0.57  & 0.91  & 0.88  & 0.09  & 0.18 \\
\midrule
\multicolumn{1}{|c|}{troponin} & \multicolumn{1}{c|}{L} & 10/60 & 0.48  & 0.42  & 0.17  & 0.16  & 0.23  & 0.22 \\
\midrule
\multicolumn{1}{|c|}{tube status} & \multicolumn{1}{c|}{VS} & 38/123 & 0.62  & 0.59  & 0.52  & 0.4   & 0.17  & 0.15 \\
\midrule
\multicolumn{1}{|c|}{vancomycin} & \multicolumn{1}{c|}{M} & 36/71 & 0.61  & 0.54  & 0.64  & 0.53  & 0.22  & 0.31 \\
\midrule
\multicolumn{1}{|c|}{vancomycin, trough} & \multicolumn{1}{c|}{L} & 13/41 & 0.46  & 0.62  & 0.37  & 0.48  & 0.28  & 0.19 \\
\midrule
\multicolumn{1}{|c|}{ventilator status} & \multicolumn{1}{c|}{VS} & 15/124 & 0.87  & 0.83  & 0.63  & 0.52  & 0.07  & 0.05 \\
\midrule
\multicolumn{1}{|c|}{white blood cell count (WBC)} & \multicolumn{1}{c|}{L} & 132/156 & 0.72  & 0.59  & 0.91  & 0.87  & 0.09  & 0.14 \\
\midrule
\multicolumn{3}{|c|}{                                                Average} & 0.7   & 0.64  & 0.62  & 0.56  & 0.16  & 0.21 \\
\bottomrule
\end{longtable}
\end{center}
\end{scriptsize}

\setcounter{figure}{0}
\renewcommand{\thefigure}{A\arabic{figure}}
\begin{figure}[!h]
    \centering
    \includegraphics[width=\textwidth]{./pictures/FigureS1}
    \caption{
    Calibration curves for per-physician models. Based on ECE values, HLR models are better calibrated than LR models for all physicians.
    }\label{fig:s1}
\end{figure}