\section{Background}\label{sec:background}
In this section, we provide brief descriptions of the LEMR system, hierarchical models, and past studies that have examined physician-related variability.

\subsection{The LEMR system}\label{sec:lemr}
The LEMR system uses a data-driven approach to prioritize patient information that is relevant in the context of a clinical task \cite{King2018,King2019}. The system uses machine learning to automatically identify and highlight relevant patient information for a specified task, for example, the task of summarizing a patient’s clinical status at morning rounds in the ICU. In ICU morning rounds, the clinical team reviews pertinent information and the status of each patient; for each patient, one team member reviews information in the EMR system and orally presents a summary of the patient’s clinical status to the team. Reviewing and identifying relevant patient information, called pre-rounding, is time-consuming and laborious. The goal of the LEMR system is to use machine learning to automatically identify and highlight the relevant information required for a given clinical task such as pre-rounding. The predictive models of the LEMR system are derived using the information-seeking behavior of physicians when they search for relevant information in the EMR in the context of the clinical task. In particular, eleven critical care physicians reviewed the EMRs of ICU patients and marked the information that was relevant to pre-rounding, and predictive models were developed from this data.

\subsection{Hierarchical models}

Hierarchical models, also known as \textit{multilevel} models, are useful in modeling hierarchically structured data because they can capture variability at different levels of the hierarchy \cite{gelman2006data}. For example, consider predicting the mortality rate in a hospital with several units, such as critical care, general medical care, and emergency care. The data has a two-level hierarchical structure with the hospital at the first level and the units at the second level of the hierarchy. The overall mortality rate at the hospital level is obtained by combining the unit-level mortality rates in some fashion. A hierarchical model explicitly estimates the variability of the mortality rates across the units and uses those estimates to derive the hospital level mortality rate, which can result in a better estimate of the overall mortality rate compared to using non-hierarchical models.

In a similar fashion, the information-seeking data used to develop the LEMR models has a two-level hierarchical structure, where the top level corresponds to data that denote the entire \textit{population} of physician reviewers and the bottom level corresponds to data that denote individual physicians. For specific patient information such as serum creatinine, its relevance is expected to differ across physician reviewers. A hierarchical model of the LEMR data explicitly captures this variability that is likely to be useful in deriving more accurate predictive models.

\subsection{Physician-related variability}
Physician-related variability in healthcare outcomes has been of interest for decades, going back to the 1970s with studies reporting the effects of geographic location on clinical outcomes such as mortality and length of stay \cite{Burns1991}. In particular, variation in individual physician characteristics and practice styles has been recognized as a source of variability in clinical outcomes after adjusting for the health status of patients and the quality of healthcare services \cite{Ruppel2020,Wilkinson2013,Yadav2019,Garland2006,Guterman2016,Obermeyer2015,Pollack2017,DeMott1990}. For example, variability in cesarean section rates has been attributed to physician practice style after controlling for patient characteristics and risk factors, status of the medical facility, and physician years of experience \cite{DeMott1990}. A study concluded that variability across individual physicians may impact the quality of preference-sensitive critical care delivery \cite{Yadav2019}. A recent study analyzed physician search patterns in the EMR and uncovered considerable variation in information-seeking behavior \cite{Ruppel2020}. In general, hierarchical modeling has been applied in various clinical settings to account for physician-related variability where the data has a hierarchical structure and can be grouped by a variety of factors such as country, state, or hospital site \cite{Wang2011,Chung2013,Pan2018,Berta2016,Berta2019,Towne2018,Lin2017}.
